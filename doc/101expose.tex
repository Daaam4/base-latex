\clearpage

\chapter{Outline Concept}

\paragraph{Introduction}
% Context: MBSE is crucial for managing the complexity of modern Cyber-Physical Systems (CPS). However, without strict guidelines, models become inconsistent and non-reusable.

The development of modern Cyber-Physical Systems (CPS) is characterized by immense complexity, a challenge managed by Model-Based Systems Engineering (MBSE). \cite{boelsen,bergemann,dehn}

This approach relies on standardized frameworks like Systems Modeling Language (SysML) to model discipline-specific subsystems in a central model. \cite{boelsen, zirui, dehn, eduardo, yildirim,friedenthal,jansen, beers}

However, the abstract nature of SysML leads to inconsistencies, non-reusable models, making modeling guidelines essential for ensuring quality, consistency, and reusability in MBSE practices. \cite{boelsen,zirui,eduardo,yildirim}

% Problem Statement: In SysML v1, modeling guidelines were typically "static" artifacts (textual documents or rigid UML profiles) that checked syntax but lacked semantic depth. SysML v2 introduces formal semantics, changing how guidelines can be implemented.

Jansen, Cibrián, Boelsen, Bergemann, Molnár, Zavada

% Research Question: How does the implementation mechanism of modeling guidelines for consistency validation differ between SysML v1 and SysML v2, and what are the implications for automated verification?

\paragraph{Theoretical Background: From UML to KerML}
% SysML v1 Foundations: Explain that v1 is a profile of UML. It relies on "Stereotypes" to add meaning, but the underlying semantics remain tied to software engineering concepts, limiting expressiveness for physical systems

% SysML v2 Foundations: Introduce the Kernel Modeling Language (KerML). Explain that v2 is grounded in formal first-order logic, enabling precise "usage-focused" modeling independent of UML.

% Definition of Modeling Guidelines: Briefly define guidelines not just as "style guides" but as technical constraints ensuring model consistency and correctness.

\paragraph{Status Quo: Guidelines in SysML v1}

% Mechanism (Profiles & OCL): Analyze how guidelines are implemented in v1. They typically involve creating a static Profile (e.g., for a specific standard like VDI/VDE 3682) and using Object Constraint Language (OCL) to enforce rules.

% Limitations:
%% Rigidity: Guidelines are "external" to the model. If the standard changes, the profile breaks.
%% Lack of Semantics: OCL checks the structure (syntax), but often misses the meaning (semantics) because v1 lacks formal grounding. %% Manual Effort: High effort is required to keep profiles consistent with domain needs.

\paragraph{The Paradigm Shift: Guidelines in SysML v2}

% Mechanism 1: Semantic Libraries:
%% Argument: In v2, guidelines are not just rules on top of the model; they are modeled as language extensions. Introduce "Semantic Libraries" where domain concepts (like statecharts or physical constraints) are defined as reusable, formal libraries.
%% Benefit: Supporting a new guideline becomes a modeling task (refining a library), not a hard-coded transformation task.

% Mechanism 2: Metamodel-Driven Validation:
%% Argument: Because v2 has a formal metamodel, validation can be automated. We can now check if a model is "semantically consistent" (meaningful) rather than just "syntactically correct" (well-formed).
%% Technique: Using the metamodel as a formal specification to derive validation rules automatically.

\paragraph{Advanced Validation Capabilities in v2}
% Constraint Propagation & Solvers:
%% Argument: SysML v2 allows mathematical analysis directly in the model. Guidelines can now define mathematical constraints (ranges, equations) that a solver (CSP/SMT) checks automatically.
%% Contrast: This moves guidelines from "formatting rules" to "mathematical correctness".

% Formal Verification:
%% Argument: Strict profiles in v2 (like "MontiBelle") can restrict the language to a subset that is fully provable by theorem provers (e.g., Isabelle), ensuring safety-critical properties.

\paragraph{Discussion: The Cost of Power}
% Complexity Trade-off: While v2 allows deeper validation, its abstract nature makes it harder for engineers to use correctly without strict guidance.

% The Need for Methodology: Just having the language isn't enough. We need specific methodologies (like "Motego") to structure the usage of v2, otherwise, reuse is impossible.

% Collaborative Challenges: In collaborative environments (Dataspaces), guidelines must ensure that model fragments exchanged between partners remain consistent.

\paragraph{Conclusion}

% Summary: The shift from v1 to v2 is a shift from Static Syntax Checking (Profiles) to Dynamic Semantic Validation (Libraries & Solvers).

% Outlook: Future modeling guidelines will likely be executable software libraries rather than PDF documents, enabling "correct-by-construction" system models.

\addtocontents{toc}{\vspace{0.8cm}}