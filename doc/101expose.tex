\clearpage

\chapter{Expose}
% Topic Definition & Relevance
Modeling guidelines are a set of rules, constraints, patterns, and heuristics used to ensure the quality, consistency, and reusability of system models. \cite{zirui} In Model-Based Systems Engineering (MBSE), guidelines are critical for managing complexity, enabling collaboration, and validating system correctness. \cite{bergemann, boelsen, eduardo, dehn, yildirim}\\
With the recent introduction of the SysML v2 standard, the entire practice of MBSE is evolving. SysML v2's new features - particularly its textual syntax , formal metamodel, and standardized API - \cite{boelsen, zirui, eduardo, friedenthal, jansen} present a paradigm shift, moving beyond the informal ``best practices`` of SysML v1  \cite{boelsen, eduardo, beers, jansen}. This paper will explore the current state of modeling guidelines in the new SysML v2 landscape.\\

% Research Question
This paper investigates the following question: \textit{How does the introduction of SysML v2 fundamentally change the landscape of modeling guidelines in MBSE, particularly regarding their formalization, validation, and enforcement?} it will explore the new opportunities for precise, automated guidelines and the new challenges that arise from SysML v2's design and adoption.

% Key Arguments & Hypothese
\begin{itemize}
    \item \textbf{Hypothesis 1 (Opportunity - Formalization):} SysML's formal metamodel and textual notation enable a new class of precise, verifiable guidelines that were not possible in SysML v1.' \cite{dehn}
    
    \item \textbf{Hypothesis 2 (Opportunity - Validation):} The v2 metamodel allows for automated semantic and syntactic consistency checking, moving guidelines from passive documents to active, verifiable parts of the model. \cite{eduardo}

    \item \textbf{Hypothesis 3 (Challenge - Complexity):} The language design of SysML v2 itself presents challenges to maintainability and portability, which complicates the development and adaptation of guideline sets. \cite{jansen}
\end{itemize}

% Source & Research Plan
The methodology will be an exploratory literature review. The primary sources will be recent (2022-2025) academic papers from IEEE, INCOSE, and relevant journals that specifically address the implementation, validation, and methodology of SysML v2.

\addtocontents{toc}{\vspace{0.8cm}}