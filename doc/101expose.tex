\clearpage

\chapter{Expose}
% Topic Definition & Relevance
The development of modern Cyber-Physical Systems (CPS) is characterized by immense complexity, a challenge managed by Model-Based Systems Engineering (MBSE). \cite{boelsen,bergemann,dehn} \\
This approach relies on standardized frameworks like Systems Modeling Language (SysML) to model discipline-specific subsystems in a central model. \cite{boelsen, zirui, dehn, eduardo, yildirim,friedenthal,jansen, beers}
However, the flexibility of SysML leads to inconsistencies, non-reusable models, making modeling guidelines essential for ensuring quality, consistency, and reusability in MBSE practices. \cite{boelsen,zirui,eduardo,yildirim} \\
The introduction of SysML v2 is intended to enhance MBSE adoption over its predecessor. \cite{friedenthal} SysML v2's new features - particularly its textual syntax , formal metamodel, and standardized API - aim at improving precision, expressiveness, usability, interoperability, and extensibility. \cite{boelsen, zirui, eduardo, friedenthal, jansen} \\

% Research Question
This paper investigates the following question: \textit{How does the introduction of SysML v2 fundamentally change the landscape of modeling guidelines in MBSE, particularly regarding their formalization, validation, and enforcement?}

% Key Arguments & Hypothese
\begin{itemize}
    \item \textbf{Hypothesis 1 (Opportunity - Formalization):} SysML's formal metamodel and textual notation enable a new class of precise, verifiable guidelines that were not possible in SysML v1.' \cite{dehn}
    
    \item \textbf{Hypothesis 2 (Opportunity - Validation):} The v2 metamodel allows for automated semantic and syntactic consistency checking, moving guidelines from passive documents to active, verifiable parts of the model. \cite{eduardo}

    \item \textbf{Hypothesis 3 (Challenge - Complexity):} The language design of SysML v2 itself presents challenges to maintainability and portability, which complicates the development and adaptation of guideline sets. \cite{jansen}
\end{itemize}

% Source & Research Plan
The methodology will be an exploratory literature review. The primary sources will be recent (2022-2025) academic papers from IEEE, INCOSE, and relevant journals that specifically address the implementation, validation, and methodology of SysML v2.

\addtocontents{toc}{\vspace{0.8cm}}