\clearpage

\paragraph{Introduction}

\begin{itemize}
    \item \textbf{Context}: Introduction to MBSE as methodology for managing CPS. MBSE requires formal models to ensure consistency, traceability, and reusability. \cite{boelsen,bergemann,dehn}
    
    \item \textbf{Introduction to SysML v1 and v2}: Briefly introduce SysML v1 industry standard that laid the groundwork for MBSE but suffers from limitations, and SysML v2 as the next-generation standard addressing these issues. \cite{boelsen, zirui, dehn, cibrian, yildirim,friedenthal,jansen, beers}
    
    \item \textbf{Problem Statement}: SysML v2 architectural overhaul overcomes v1 limitations, its abstract nature introduces a 'flexibility paradox' resulting in model inconsistencies. This semantic amiguity creates new problems for automated verification, necessitating a change in the implementation mechanism of modeling guidelines. \cite{boelsen, friedenthal, cibrian,zirui}
    
    \item \textbf{Research Question}: How does the implementation mechanism of modeling guidelines for consistency validation differ between SysML v1 and SysML v2, and what are the implications for automated verification?
    
    \item \textbf{Thesis}: The change in SysML v2 architecture, which is grounded in formal axiomatic semantics, enables a significant increase in the rigor, automation of automated verification. \cite{molnar,ratzke}
\end{itemize}

\paragraph{Theoretical Background: From UML to KerML}

\begin{itemize}
    \item \textbf{SysML v1 Foundations}: Detail SysML v1's UML foundation and it's constraints in expressiveness and scalability. \cite{cibrian,jansen}
    \begin{itemize}
        \item Modeling guidelines were achieved by defining Domain Specific Modelings Languages (DSML) as UML Profiles via stereotypes. \cite{beers,boelsen}

        \item V1 Constraint Mechanism: V1 relied of Object Constraint Language (OCL) to enforce rules on these profiles. \cite{beers,ratzke}
    \end{itemize}

    \item \textbf{SysML v2 Foundations}: Introduce the Kernel Modeling Language (KerML), which provides foundational base concepts and formal semantics for SysML rooted in first-order logic axioms (FOL is the standard for the formalization of mathematics into axioms). \cite{jansen}
    \begin{itemize}
        \item V2 Constraint Mechanism: V2 natively support constraints modeling via KerML-based constructs (invariant, assert, requirement). \cite{molnar,ratzke}
        \item V2 interoperability and guidelines: Graphical and Textual notations, the standardized API for tool interoperability and standarized exchange via JSON. \cite{friedenthal,cibrian}
        \item Guidelines can be implemented using semantic metadata and model libraries without replacing the complexity of profiles/stereotypes. \cite{boelsen,molnar}
    \end{itemize}

\end{itemize}

% Start results in new page
\clearpage

\paragraph{Results - The Implementation Mechanism Difference}

\begin{itemize}
    \item \textbf{V1 implementation}: External Rule Application (Monitoring/Rule-Checking)
    \begin{itemize}
        \item SysML v1 consistency was achieved by adopting the rule checking approach. \cite{bergemann}
        
        \item The definition of guidelines required manual or customized tools (Custom OCL contraints, Model Checkers) to validate models against the defined rules.  \cite{bergemann}
        
        \item These approach often lack percision due to the lack of formal semantics in SysML v1. \cite{jansen}
    \end{itemize}

    \item \textbf{V2 Implementation}: V2 implements consistency through a systematic, metamodel-based validation approach. \cite{cibrian}
    \begin{itemize}
        \item The metamodel functions as formal specification from which validation rules are directly derived ensuring automated detection of structural and semantic inconsistencies. \cite{cibrian}
        
        \item Guideline Enforcement: Guidelines are enforced directly using the native notation and syntax combined with metadata definitions. \cite{boelsen}
    \end{itemize}

    \item \textbf{Direct vs. Indirect Validation}: Detail the difference between validation process in V1 where it required mapping high-level concepts to OCL expressions and transforming models to support external checking, whereas V2 validation operates directly on the formal semantics of the model via its abstract syntax representation. \cite{zavada, cibrian}
\end{itemize}




\paragraph{Discussion and Implications}

\begin{itemize}
    \item \textbf{Formal Methods (FM) Integration}: foundation of V2 drastically simplifies the application of FM. Verification tools like Gamma, Imandra, SAVVS, and SysMD are being integrated to leverage V2's features (e.g., the standardized API) \cite{molnar, ratzke}

    \item \textbf{Advanced Behavioral Verification}: V2 enables formal verification of behavioral models (state machines) by semantically mapping them to theorem prover encodings. \cite{kausch}
    
    \item \textbf{Constraint Satisfaction and Propagation}: The formal constraint capabilities in SysML v2 enable powerful constructive model analysis. This is achieved by reformulating the system model as a Constraint Satisfaction Problem (CSP) and applying Constraint Propagation algorithms to analyze requirements and reduce variable domains. \cite{molnar, ratzke}

    \item \textbf{Verification Scalability}: In contrast to potentially exponential complexity of state generation in some manual V1 checks, V2 benefits from underlying formalisms (like Focus) that support the compositionality of refinement. \cite{dehn, kausch}
    
    \item \textbf{Implications for Informal Modeling} Discuss how V2's formalism can support flexible modeling techniques (like transforming structured natural language templates into formal V2 code) when guided by LLMs and precise mapping, thereby lowering the complexity barrier for domain experts. \cite{dukic}
\end{itemize}

\addtocontents{toc}{\vspace{0.8cm}}