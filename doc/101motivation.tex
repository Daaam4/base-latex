\clearpage

\chapter{Motivation}
Das Gas-Wasser-Strom-Problem gehört zu den bekanntesten Klassikern der graphentheoretischen Unterhaltungsmathematik. Es wurde im Jahr 1917 von Henry Ernest Dudeney in seinem Buch ``\textit{Amusements in Mathematics}`` vorgestellt. Die Aufgabenstellung ist scheinbar einfach: Drei Häuser sollen jeweils unabhängig voneinander mit den drei Versorgungseinheiten Gas, Wasser und Strom verbunden werden. Dabei dürfen sich keine der neun Verbindungsleitungen kreuzen.\cite{buesing2010graphen}\\

Das Problem illustriert die Fragestellung, ob sich bestimmte Netzwerke in der Ebene so darstellen lassen, dass ihre Kanten einander nicht schneiden.\\
Die Relevanz dieses Problems zeigt sich in zahlreichen technischen, wissenschaftlichen und praktischen Anwendungen, in denen die Vermeidung von Kreuzungen eine entscheidende Rolle spielt.\\

Bei der Visualisierung von Graphen, etwa in der Informatik oder Netzwerkanalyse, erleichtert eine Darstellung ohne sich kreuzende Kanten das Verständnis komplexer Strukturen. So lassen sich Muster, Zusammenhänge oder zentrale Elemente besser erkennen.\\

Auch in der Elektrotechnik etwa ist es essentiell, auf einem Mikrochip die Leiterbahnen zwischen elektronischen Komponenten wie Widerständen so zu platzieren, dass es zu keinen unerwünschten Überschneidungen und damit verbundenen Kurzschlüssen kommt.\\

\begin{figure}[htbp]
    \centering
    \includegraphics[width=0.7\textwidth]{./pic/3_utilities_problem_plane.png}
    \caption{Darstellung des Gas-Wasser-Strom-Problems in der Ebene. Quelle: \cite{wiki-utilities}}
    \label{fig:gas-wasser-strom-problem}
\end{figure}

\addtocontents{toc}{\vspace{0.8cm}}