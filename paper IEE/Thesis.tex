\documentclass[conference]{IEEEtran}
\IEEEoverridecommandlockouts
% The preceding line is only needed to identify funding in the first footnote. If that is unneeded, please comment it out.
\usepackage{cite}
\usepackage{amsmath,amssymb,amsfonts}
\usepackage{algorithmic}
\usepackage{graphicx}
\usepackage{textcomp}
\usepackage{xcolor}
\def\BibTeX{{\rm B\kern-.05em{\sc i\kern-.025em b}\kern-.08em
    T\kern-.1667em\lower.7ex\hbox{E}\kern-.125emX}}
\begin{document}

%
% Paper Title
%
\title{Conference Paper Title*\\
{\footnotesize \textsuperscript{*}Note: Sub-titles are not captured in Xplore and
should not be used}
\thanks{Identify applicable funding agency here. If none, delete this.}
}

\author{\IEEEauthorblockN{1\textsuperscript{st} Nassim Awabdy}
\IEEEauthorblockA{\textit{Fachbereich 5} \\
\textit{Aachen University of Applied Sciences}\\
Aachen, Germany \\
nassim.awabdy@alumni.fh-aachen.de}
}

\maketitle

%
% Abstract
%
\begin{abstract}
This document is a model and instructions for \LaTeX.
This and the IEEEtran.cls file define the components of your paper [title, text, heads, etc.]. *CRITICAL: Do Not Use Symbols, Special Characters, Footnotes, 
or Math in Paper Title or Abstract.
\end{abstract}

%
% Keywords
%
\begin{IEEEkeywords}
component, formatting, style, styling, insert
\end{IEEEkeywords}

%
% Introduction 
%
\section{Introduction}
% Context (MBSE, CPS)
% - Define the difference between verification and validation 
% 
% - Methodology of the thesis
Model-Based Systems Engineering (MBSE) has become a cornerstone methodology for managing the complexity of modern Cyber-Physical Systems (CPS). The development of CPSs is inherently challenging because they combine software with hardware embedded in the physical world. MBSE has become the industry standard for the development and design management of these complex systems, extending the classical systems engineering by utilizing a centralized system model.

By emphasizing the use of \textbf{formal models} throughout the system lifecycle, MBSE supports the design, analysis, and verification of system representations, promoting consistency, traceability, and reusability across engineering processes. MBSE enables system architects to respond more quickly and effectively to numerous changes in requirements that occur during the development process.

In this context, the Models are crucial for specifying the high-level, architecture, functionality, uses cases, requirements, and constraints of the technical systems.

% Problem Statement (SysML v1 as a standard, its limitations, and SysML v2 as a solution)
+ Problem statement (Introduction to SysML v1 and V2)

% Highlight the difference between validation and verification

% Research Question & Thesis Statement
+ Research question
+ Highlight the focus which is on specific scenarios not a holistic overview of the tools
+ Thesis statement

%
% Theoretical Background
%
\section{Theoretical Background}
\subsection{MBSE}
\subsection{SysML v1 Foundations}
\subsection{SysML v2 Foundations}

%
% Results
%
\section{Results}
\subsection{SysML v1 }

%
% Discuss & Implications
%
\section{Discussion and Implications}

%
% Conclusion & Outlook
%
\section{Conclusion and Outlook}

\clearpage



%
% Acknowledgment
%
\section*{Acknowledgment}

%
% Bibliography
%
\begin{thebibliography}{00}
\bibitem{b1} G. Eason, B. Noble, and I. N. Sneddon, ``On certain integrals of Lipschitz-Hankel type involving products of Bessel functions,'' Phil. Trans. Roy. Soc. London, vol. A247, pp. 529--551, April 1955.
\bibitem{b2} J. Clerk Maxwell, A Treatise on Electricity and Magnetism, 3rd ed., vol. 2. Oxford: Clarendon, 1892, pp.68--73.
\bibitem{b3} I. S. Jacobs and C. P. Bean, ``Fine particles, thin films and exchange anisotropy,'' in Magnetism, vol. III, G. T. Rado and H. Suhl, Eds. New York: Academic, 1963, pp. 271--350.
\bibitem{b4} K. Elissa, ``Title of paper if known,'' unpublished.
\bibitem{b5} R. Nicole, ``Title of paper with only first word capitalized,'' J. Name Stand. Abbrev., in press.
\bibitem{b6} Y. Yorozu, M. Hirano, K. Oka, and Y. Tagawa, ``Electron spectroscopy studies on magneto-optical media and plastic substrate interface,'' IEEE Transl. J. Magn. Japan, vol. 2, pp. 740--741, August 1987 [Digests 9th Annual Conf. Magnetics Japan, p. 301, 1982].
\bibitem{b7} M. Young, The Technical Writer's Handbook. Mill Valley, CA: University Science, 1989.
\end{thebibliography}

\end{document}
