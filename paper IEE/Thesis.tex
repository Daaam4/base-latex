\documentclass[conference]{IEEEtran}
\IEEEoverridecommandlockouts
% The preceding line is only needed to identify funding in the first footnote. If that is unneeded, please comment it out.
\usepackage{cite}
\usepackage{amsmath,amssymb,amsfonts}
\usepackage{algorithmic}
\usepackage{graphicx}
\usepackage{textcomp}
\usepackage{xcolor}
\def\BibTeX{{\rm B\kern-.05em{\sc i\kern-.025em b}\kern-.08em
    T\kern-.1667em\lower.7ex\hbox{E}\kern-.125emX}}
\begin{document}

%
% Paper Title
%
\title{Conference Paper Title*\\
{\footnotesize \textsuperscript{*}Note: Sub-titles are not captured in Xplore and
should not be used}
\thanks{Identify applicable funding agency here. If none, delete this.}
}

\author{\IEEEauthorblockN{1\textsuperscript{st} Nassim Awabdy}
\IEEEauthorblockA{\textit{Fachbereich 5} \\
\textit{Aachen University of Applied Sciences}\\
Aachen, Germany \\
nassim.awabdy@alumni.fh-aachen.de}
}

\maketitle

%
% Abstract
%
\begin{abstract}
This document is a model and instructions for \LaTeX.
This and the IEEEtran.cls file define the components of your paper [title, text, heads, etc.]. *CRITICAL: Do Not Use Symbols, Special Characters, Footnotes, 
or Math in Paper Title or Abstract.
\end{abstract}

%
% Keywords
%
\begin{IEEEkeywords}
component, formatting, style, styling, insert
\end{IEEEkeywords}

%
% Introduction 
%
\section{Introduction}
% Context - MBSE, CPS
Modern Cyber-Physical Systems (CPS) are technical systems that combine mechanical, electronic, and software subsystems with physical elements embedded in the real world. 

The development of CPSs is becoming increasingly complex and challenging, due to their interdisciplinary nature and the need to ensure seamless integration between their physical and computational components. \cite{boelsen}

Model-Based Systems Engineering (MBSE)

Model-Based Systems Engineering (MBSE) has become a cornerstone methodology for managing the complexity of modern Cyber-Physical Systems (CPS). The development of CPSs is inherently challenging because they combine software with hardware embedded in the physical world. MBSE has become the industry standard for the development and design management of these complex systems, extending the classical systems engineering by utilizing a centralized system model. \cite{beers}

By emphasizing the use of \textbf{formal models} throughout the system lifecycle, MBSE supports the design, analysis, and verification of system representations, promoting consistency, traceability, and reusability across engineering processes. MBSE enables system architects to respond more quickly and effectively to numerous changes in requirements that occur during the development process.

In this context, the Models are crucial for specifying the high-level, architecture, functionality, uses cases, requirements, and constraints of the technical systems.

% - Define the difference between verification and validation 
% 
% - Methodology of the thesis


% Problem Statement (SysML v1 as a standard, its limitations, and SysML v2 as a solution)


% Research Question & Thesis Statement
% + Highlight the focus which is on specific scenarios not a holistic overview of the tools

% Highlight the difference between validation and verification

%
% Theoretical Background
%
\section{Theoretical Background}
\subsection{MBSE}
\subsection{SysML v1 Foundations}
\subsection{SysML v2 Foundations}

%
% Results
%
\section{Results}
\subsection{SysML v1 Implementation of Modeling Guidelines}
\subsection{SysML v2 Implementation of Modeling Guidelines}
\subsection{Comparative Analysis}

%
% Discuss & Implications
%
\section{Discussion and Implications}

%
% Conclusion & Outlook
%
\section{Conclusion and Outlook}

%
% Acknowledgment
%
\section*{Acknowledgment}

%
% Bibliography
%
\begin{thebibliography}{00}

\bibitem{beers} L. Beers, H. Nabizada, M. Weigand, F. Gehlhoff, and A. Fay, “A sysml profile for the standardized description of processes during system development,” in 2024 IEEE International Systems Conference (SysCon).
IEEE, 2024, pp. 1--8.

\bibitem{bergemann} S. Bergemann, “Challenges in multi-view model consistency management for systems engineering,” in Modellierung 2022 Satellite Events. Bonn: Gesellschaft f¨ur Informatik e.V., 2022, pp. 77--89.

\bibitem{boelsen} K. Boelsen, M. May, G. Jacobs, and et al., “Sysml v2 based modelling guidelines for mechanical system elements,” Forsch Ingenieurwes, vol. 89, no. 60, 2025.

\bibitem{cibrian} E. Cibri\'{a}n, J. Olivert-Iserte, C. D\'{i}ez-Fenoy, R. Mendieta, J. Llorens, and J. M. \'Alvarez Rodr\'{i}guez, “Ensuring semantic consistency in sysml v2 models through metamodel-driven validation,” IEEE Access, vol. 13, pp. 121 444--121 457, 2025.

\bibitem{dehn} S. Dehn, G. Jacobs, P. H\"ock, and G. H\"opfner, “Enhancing model-based development with formalized requirements: integrating temporal logic and sysml v2 for comprehensive state and transition modeling,” Forschung im Ingenieurwesen, vol. 89, no. 1, p. 53, 2025.

\bibitem{dukic} P. Duki\'c, “Generating sysmlv2 models from structured natural language,” 2025.

\bibitem{friedenthal}

\bibitem{jansen} N. Jansen, J. Pfeiffer, B. Rumpe, D. Schmalzing, and A. Wortmann, “The language of sysml v2 under the magnifying glass.” J. Object Technol., vol. 21, no. 3, pp. 3--1, 2022.

\bibitem{kausch} H. Kausch, M. Pfeiffer, D. Raco, B. Rumpe, and A. Schweiger, “Model-driven development for functional correctness of avionics systems: a verification framework for sysml specifications,” CEAS Aeronautical Journal, vol. 16, no. 1, pp. 33--48, 2025.

\bibitem{molnar} V. Moln\'{a}r, B. Graics, A. V\"or\"os, S. Tonetta, L. Cristoforetti, G. Kimberly,
P. Dyer, K. Giammarco, M. Koethe, J. Hester et al., “Towards the formal
verification of sysml v2 models,” in Proceedings of the ACM/IEEE 27th
International Conference on Model Driven Engineering Languages and
Systems, 2024, pp. 1086--1095.

\bibitem{raedler}

\bibitem{ratzke} A. Ratzke, J. Koch, and C. Grimm, “Modeling and analysis of system models with constraints in sysml v2,” in 2025 20th Annual System of Systems Engineering Conference (SoSE), 2025, pp. 1--6.

\bibitem{yildirim}

\bibitem{zavada}

\bibitem{zirui}

\end{thebibliography}

\end{document}
