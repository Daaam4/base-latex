\documentclass[conference]{IEEEtran}
\IEEEoverridecommandlockouts
% The preceding line is only needed to identify funding in the first footnote. If that is unneeded, please comment it out.
\usepackage{cite}
\usepackage{amsmath,amssymb,amsfonts}
\usepackage{algorithmic}
\usepackage{graphicx}
\usepackage{textcomp}
\usepackage{xcolor}
\def\BibTeX{{\rm B\kern-.05em{\sc i\kern-.025em b}\kern-.08em
    T\kern-.1667em\lower.7ex\hbox{E}\kern-.125emX}}
\begin{document}

%
% Paper Title
%
\title{Mechanisms for Model Consistency: A Comparative Analysis of Guideline Implementation in SysML v1 and SysML v2 }

\author{\IEEEauthorblockN{1\textsuperscript{st} Nassim Awabdy}
\IEEEauthorblockA{\textit{Fachbereich 5} \\
\textit{Aachen University of Applied Sciences}\\
Aachen, Germany \\
nassim.awabdy@alumni.fh-aachen.de}
}

\maketitle

%
% Abstract
%
\begin{abstract}
TODO
\end{abstract}

%
% Keywords
%
\begin{IEEEkeywords}
cyber-physical systems, model-based systems engineering, sysml, modeling guidelines, model validation, model verification, systems engineering
\end{IEEEkeywords}

%
% Introduction 
%
\section{Introduction}
%% CPS
Modern Cyber-Physical Systems (CPS) are technical systems that combine mechanical, electronic, and software subsystems with physical elements embedded in the real world.
The development of CPSs is becoming increasingly complex and challenging, due to their interdisciplinary nature and the need to ensure seamless integration between their physical and computational components \cite{bergemann,boelsen}.

%% MBSE
Model-Based Systems Engineering (MBSE) is a methodology for the development and management such complex systems, that addresses issues arising from the complexity and interdisciplinary nature of CPS, and provides the agility required to adapt to changing requirements and technologies. 
MBSE incorporates a centralized system model as the primary source of information, throughout the system lifecycle \cite{bergemann,dehn,cibrian,dukic,boelsen,friedenthal}. 

%% (Status Quo) SysML v1 and its limitations 
SysML v1 has been widely adopted as the standard for modelling CPS and served as a key enabler MBSE. SysML v1 is a graphical, general purpose modelling language that is defined as an extension of the Unified Modeling Language (UML). Because it was built on top of UML, SysML v1 inherited several limitations from UML, that limited it's expressiveness and usability for CPS modelling. However, it still provided a solid  foundation for specifying and analyzing a systems's behaviour, structure and requirements \cite{boelsen,cibrian,friedenthal}.
 
%% (Shift) SysML v2 
The release of SysML v2 represents the next generation of the Systems Modeling Language, designed as an overhaul of SysML v1 that addresses its limitations and enhance the efficacy of MBSE practices.
Unlike it's predecessor, SysML v2 is built upon the Kernel Modeling Language (KerML), this approach ensures that SysML v2 inherits a formal semantic foundation, that is crucial for enhanced precision and automation in MBSE workflows.
\cite{boelsen,cibrian,friedenthal,molnar}.

%% Problem Statement (SysML v1 as a standard, its limitations, and SysML v2 as a solution)
While these advancements introduced by SysML v2 are promising, the abstract nature of the language still presents challenges for ensuring consistent modeling practices across diverse engineering teams.
Model inconsistencies creates a high risk of redundant effort, potential modeling errors, and lack of reuse of system elements, preventing them from being aggregated into a coherent overall system model.
Therefore, mechanisms for implementing and enforcing modeling guidelines must evolve to leverage the native formal capabilities of SysML v2, that enable effective model verification processes throughout the development lifecycle \cite{boelsen,bergemann,cibrian,molnar,ratzke}.

%% Outline the difference between verification and validation
Within the context of model analysis, a distinct differentiation between model validation and model verification is necessary. 

Model \textit{validation} is a form of static analysis that ensures a model's structural and semantic correctness. By detecting inconsistencies ranging from syntactic errors to mismatched data types, validation guarantees the model's precision and coherence. This process acts as a critical quality control, ensuring the model is reliable before it undergoes transformation or dynamic verification \cite{cibrian, zavada}.

Model \textit{verification}, conversely, is a dynamic analysis technique focused on ensuring that the system specified by the model behaves exactly as intended under various conditions, thereby confirming behavioral adherence to specifications \cite{zavada}.

%% Research Question
This work addresses the following research question: \textbf{How do the mechanisms for implementing modeling guidelines differ between SysML v1 and SysML v2, and how do these differences impact the capabilities for model validation and automated verification?}

%% Methodology & Scope
To answer this, we conduct a comparative analysis of the underlying implementation mechanisms, specifically contrasting the profile-based constraints of SysML v1 with the metamodel-driven and formal semantic capabilities of SysML v2. 
We analyze the implications of this shift by examining specific scenarios where these mechanisms facilitate structural validation \cite{cibrian} and dynamic verification \cite{dehn, molnar}, rather than providing a holistic overview of all available commercial tools.

%
% Theoretical Background
%
\section{Theoretical Background}
% MBSE - Foundations
\subsection{Model-Based Systems Engineering (MBSE)}
Model-Based Systems Engineering (MBSE) is defined by the International Council on Systems Engineering (INCOSE) as ''The formalized application of modeling to support system requirements, design, analysis, verification, and validation activities beginning in the concept stage and continuing throughout development and later life cycle stages.'' \cite{incose}. 
Unlike traditional document-based approaches, where system's information is scattered across various files, MBSE is a methodology the centralizes system's information within a unified model, which serves as the primary source of information that is managed throughout the system lifecycle \cite{friedenthal}.
This approach enhances collaboration among multidisciplinary teams, reduces errors from manual synchronization, and enables system architects to respond more quickly and effectively to numerous changes in requirements that occur during the development process \cite{dukic,yildirim,beers}.

% SysML v1 Foundations
\subsection{SysML v1 Foundations}
SysML v1 is a graphical, general purpose modeling language that is widely recognized as the standard language for MBSE, serving as a foundational tool for specifying, analyzing, designing and verifying complex multidisciplinary systems \cite{beers,cibrian,jansen}.
SysML v1, adopted by the Object Management Group (OMG) in 2007, was essential in advancing MBSE practice by providing capabilities for formally capturing system requirements, structure, behaviour, and parametrics \cite{friedenthal}.
The language is defined as an extension of the Unified Modeling Language (UML), allowing it to adopt established modeling concepts \cite{cibrian}.

%% Structure & Diagrams
SysML v1 organizes its modeling constructs into four main categories:
\begin{itemize}
    \item \textbf{Structure}: This aspect is modeled using Block Definition Diagrams (BDDs) and Internal Block Diagrams (IBDs). BBDs are used to define components, interfaces, and relationships at black-box level, while IBDs offer a white-box perspective by outlining the internal structure of a single block \cite{jansen}.
    \item \textbf{Behaviour}: SysML v1 provides several behavioral diagrams, including State Machine Diagrams, Sequence Diagrams, and Activity Diagrams \cite{jansen,yildirim}. These diagrams capture distinct uses for modeling system behaviour.
    \item \textbf{Requirements}: System requirements are specified in Requirement Diagrams, that largely rely on natural language representation, although they can be linked logically to other model elements \cite{jansen,molnar}.
    \item \textbf{Parametrics}: Parametric Diagrams define mathematical constraints and equations between system elements. They support preliminary calculations and analysis within the model, though they often require external tools for complex evaluations \cite{jansen}.
\end{itemize}

%% Stereotypes and Profiling
SysML v1 allows the restriction of modeling practices through UML profiling mechanism, enabling the construction of specialized extensions. The central element for customization is the \textbf{stereotype}, which functions as a distinct metaclass within UML \cite{beers}. Stereotypes enable the customization of existing metaclasses by associating them with specific properties and constraints, thereby tailoring the modeling language to meet domain-specific requirements \cite{beers,raedler}.

%% Object Constraint Language (OCL)
The \textbf{Object Constraint Language (OCL)} function as a specialized textual specifications language that enables formally defining rules, constraints and queries on models especially those created using UML \cite{ratzke}. It was developed to overcome the limitations of graphical modeling by allowing precise specification of rules that cannot be easily represented visually \cite{beers}.

% SysML v2 Foundations (molnar,cibrian,ratzke)
\subsection{SysML v2 Foundations}
SysML v2 represents a major evolution over it's predecessor, having been engineered independently from UML to overcome the limitations inherited from it \cite{cibrian,friedenthal}. It aims to enhance MBSE adoption and effectiveness by focusing on improving precision, expressiveness, consistency, usability, interoperability, and extensibility. \cite{friedenthal}
The foundation of SysML v2 is built upon a new general-purpose modeling language called the \textbf{Kernel Modeling Language (KerML)} \cite{molnar}.

%% KerML Overview
The mechanism of KerML is built upon a hierarchical, three-layered architecture, successively progressing from general to specific constructs \cite{omg}.

\begin{itemize}
    \item The \textbf{Root Layer} establishes the essential syntactic scaffolding for constructing models. The main focus of this layer is to define organizational constructs, such as \textit{Elements}, \textit{Namespaces}, and \textit{Relationships}, leaving out model-level semantic interpretation relative to the modeled system \cite{omg,molnar}.
    \item The \textbf{Core Layer} introduces the language's semantic foundation based on classification, defined in first-order logic axioms. The central primitive is \textit{Type}, which is divided into \textit{Classifiers}, \textit{Features}. It also defines key relationships necessary for organizing classification hierachies \cite{omg,molnar}. 
    \item The \textbf{Kernel Layer} finalizes the language specification by adding specialized constructs used in common modeling applications, such as \textit{Data Types}, \textit{Classes}, \textit{Structures}, and \textit{Behaviours} \cite{omg,molnar}.
\end{itemize}

%% Semantic Foundation
KernML achieves its consistent semantics through formal mathematical logic and library-based ontological modeling that maintain a precise interpretation of complex models.
Since the semantics in the \textit{Core Layer} are defined using first-order logic, a consistent basis for mathematical reasoning about models is established \cite{omg,molnar}.
For comprehensive concepts introduced in the \textit{Kernel Layer}, KerML extends its semantics through the reuse of elements found in the \textbf{Kernel Semantic Library}. This library is itself expressed in KerML, meaning that all concepts in the language are ultimately grounded in the same formal semantic framework \cite{omg,molnar}.

%% Key Language Features, Notations & Constructs
SysML v2 introduces several key features for further enhancing modeling capabilities, accessibility, and tool integration.
\begin{itemize}
    \item \textbf{Textual Notations}: In addition to graphical notation, SysML v2 features a standardized textual syntax that provides advantages for interoperability with external tools and exchange of models \cite{omg,molnar}.
    \item \textbf{Standardized API}: The language includes the new Systems Modeling API (SysML API), which enables full access to the model and general Model as Code workflows \cite{omg,molnar}.
    \item \textbf{Formal Requirements}: Requirements in SysML v2, are constructs that can formally specify a \textit{constraint}, declare their \textit{subject}, and have defined \textit{attributes}, moving beyond natural language representation \cite{omg,molnar}.
    \item \textbf{Cases}: SysML v2 introduced a generic \textit{Case} construct which is essentially a calculation that can declare a subject and an objective to provide a formal and executable way to check model correctness and evaluate system properties. Two specialized cases are provided \textit{Analysis Case} (Quantative Analysis) and \textit{Verification Case} (Qualitative Analysis), further enhacing model analysis capabilities \cite{molnar}.
\end{itemize}


%
% Results
%
\section{Results}

%% SysML v1 Implementation of Modeling Guidelines (beers)
\subsection{SysML v1 Implementation of Modeling Guidelines}

\begin{itemize}
    \item \textbf{mechanism}: Guidelines are implemented using UML profiling mechanism and Stereotypes
    \item \textbf{Constraint}: OCL is used to formally define rules and constraints on models
    \item \textbf{Case Study}: \textit{Beers} demonstrate that implementing a guideline (like VDI/VDE 3682) in v1 requires building a heavyweight Domain-Specific Modeling Language (DSML) profile
    \item \textbf{Limitations}: The validation is often tool-dependent (e.g., MagicDraw/Cameo specific) and heavy to maintain.
\end{itemize}

\subsection{SysML v2 Implementation of Modeling Guidelines}


\subsection{Comparative Analysis}

\clearpage

%
% Discuss & Implications
%
\section{Discussion and Implications}

%
% Conclusion & Outlook
%
\section{Conclusion and Outlook}

%
% Acknowledgment
%
\section*{Acknowledgment}

%
% Bibliography
%
\begin{thebibliography}{00}
    \bibitem{beers} L. Beers, H. Nabizada, M. Weigand, F. Gehlhoff, and A. Fay, “A sysml profile for the standardized description of processes during system development,” in 2024 IEEE International Systems Conference (SysCon). IEEE, 2024, pp. 1--8.
    
    \bibitem{bergemann} S. Bergemann, “Challenges in multi-view model consistency management for systems engineering,” in Modellierung 2022 Satellite Events. Bonn: Gesellschaft f¨ur Informatik e.V., 2022, pp. 77--89.

    \bibitem{boelsen} K. Boelsen, M. May, G. Jacobs, and et al., “Sysml v2 based modelling guidelines for mechanical system elements,” Forsch Ingenieurwes, vol. 89, no. 60, 2025.
    
    \bibitem{cibrian} E. Cibri\'{a}n, J. Olivert-Iserte, C. D\'{i}ez-Fenoy, R. Mendieta, J. Llorens, and J. M. \'Alvarez Rodr\'{i}guez, “Ensuring semantic consistency in sysml v2 models through metamodel-driven validation,” IEEE Access, vol. 13, pp. 121 444--121 457, 2025.
    
    \bibitem{dehn} S. Dehn, G. Jacobs, P. H\"ock, and G. H\"opfner, “Enhancing model-based development with formalized requirements: integrating temporal logic and sysml v2 for comprehensive state and transition modeling,” Forschung im Ingenieurwesen, vol. 89, no. 1, p. 53, 2025.
    
    \bibitem{dukic} P. Duki\'c, “Generating sysmlv2 models from structured natural language,” 2025.

    \bibitem{friedenthal} S. Friedenthal, “Future directions for mbse with sysml v2.” in MODEL-SWARD, 2023, pp. 5--9.

    \bibitem{incose} D. D. Walden, G. J. Roedler, K. J. Forsberg, R. D. Hamelin, and T. M. Shortell, Eds., \textit{INCOSE Systems Engineering Handbook: A Guide for System Life Cycle Processes and Activities}, 4th ed. Hoboken, NJ, USA: John Wiley \& Sons, 2015.
    
    \bibitem{jansen} N. Jansen, J. Pfeiffer, B. Rumpe, D. Schmalzing, and A. Wortmann, “The language of sysml v2 under the magnifying glass.” J. Object Technol., vol. 21, no. 3, pp. 3--1, 2022.
    
    \bibitem{kausch} H. Kausch, M. Pfeiffer, D. Raco, B. Rumpe, and A. Schweiger, “Model-driven development for functional correctness of avionics systems: a verification framework for sysml specifications,” CEAS Aeronautical Journal, vol. 16, no. 1, pp. 33--48, 2025.
    
    \bibitem{molnar} V. Moln\'{a}r, B. Graics, A. V\"or\"os, S. Tonetta, L. Cristoforetti, G. Kimberly, P. Dyer, K. Giammarco, M. Koethe, J. Hester et al., “Towards the formal verification of sysml v2 models,” in Proceedings of the ACM/IEEE 27th International Conference on Model Driven Engineering Languages and Systems, 2024, pp. 1086--1095.

    \bibitem{omg} Object Management Group, “Kernel Modeling Language (KerML), Version 1.0,” Object Management Group, Specification formal/25-09-01, Sep. 2025. [Online]. Available: https://www.omg.org/spec/KerML/1.0/PDF
    
    \bibitem{ratzke} A. Ratzke, J. Koch, and C. Grimm, “Modeling and analysis of system models with constraints in sysml v2,” in 2025 20th Annual System of Systems Engineering Conference (SoSE), 2025, pp. 1--6.
    
    \bibitem{raedler} S. Raedler, J. Mangler, and S. Rinderle-Ma, “Model-driven engineering method to support the formalization of machine learning using sysml,” arXiv preprint arXiv:2307.04495, 2023.

    \bibitem{yildirim}  U. Yildirim, F. Campean, A. Korsunovs, and A. Doikin, “Flow heuristics for functional modelling in model-based systems engineering,” Proceedings of the Design Society, vol. 3, pp. 1895--1904, 2023.
    
    \bibitem{zavada} Zavada, G. Kulcs\'{a}r, V. Moln\'{a}r, and Horv\'{a}th, “Towards a configurable verification and validation framework for critical cyber-physical systems,” in 2025 IEEE 8th International Conference on Industrial Cyber-Physical Systems (ICPS), 2025, pp. 1--6.
    
    \bibitem{zirui} Z. Li, F. Faheem, and S. Husung, “Collaborative model-based systems engineering using dataspaces and sysml v2,” Systems, vol. 12, no. 1, 2024. [Online]. Available: https://www.mdpi.com/2079-8954/12/1/18

\end{thebibliography}

\end{document}
