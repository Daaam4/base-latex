\documentclass[conference]{IEEEtran}
\IEEEoverridecommandlockouts
% The preceding line is only needed to identify funding in the first footnote. If that is unneeded, please comment it out.
\usepackage{cite}
\usepackage{amsmath,amssymb,amsfonts}
\usepackage{algorithmic}
\usepackage{graphicx}
\usepackage{textcomp}
\usepackage{xcolor}
\def\BibTeX{{\rm B\kern-.05em{\sc i\kern-.025em b}\kern-.08em
    T\kern-.1667em\lower.7ex\hbox{E}\kern-.125emX}}
\begin{document}

%
% Paper Title
%
\title{Conference Paper Title*\\
{\footnotesize \textsuperscript{*}Note: Sub-titles are not captured in Xplore and
should not be used}
\thanks{Identify applicable funding agency here. If none, delete this.}
}

\author{\IEEEauthorblockN{1\textsuperscript{st} Nassim Awabdy}
\IEEEauthorblockA{\textit{Fachbereich 5} \\
\textit{Aachen University of Applied Sciences}\\
Aachen, Germany \\
nassim.awabdy@alumni.fh-aachen.de}
}

\maketitle

%
% Abstract
%
\begin{abstract}
This document is a model and instructions for \LaTeX.
This and the IEEEtran.cls file define the components of your paper [title, text, heads, etc.]. *CRITICAL: Do Not Use Symbols, Special Characters, Footnotes, 
or Math in Paper Title or Abstract.
\end{abstract}

%
% Keywords
%
\begin{IEEEkeywords}
cyber-physical systems, mbse, sysml, modeling guidelines
\end{IEEEkeywords}

%
% Introduction 
%
\section{Introduction}
% CPS
Modern Cyber-Physical Systems (CPS) are technical systems that combine mechanical, electronic, and software subsystems with physical elements embedded in the real world.
The development of CPSs is becoming increasingly complex and challenging, due to their interdisciplinary nature and the need to ensure seamless integration between their physical and computational components \cite{bergemann,boelsen}.

% MBSE
Model-Based Systems Engineering (MBSE) is a methodology for the development and management such complex systems, that addresses issues arising from the complexity and interdisciplinary nature of CPS, and provides the agility required to adapt to changing requirements and technologies. 
MBSE incorporates a centralized system model as the primary source of information, throughout the system lifecycle \cite{bergemann,dehn,cibrian,dukic,boelsen,friedenthal,ratzke,yildirim,zavada,zirui}. 

% (Status Quo) SysML v1 and its limitations 
SysML v1 has been widely adopted as the standard for modelling CPS and served as a key enabler MBSE. SysML v1 is a graphical, general purpose modelling language that is defined as an extension of the Unified Modeling Language (UML). Because it was built on top of UML, SysML v1 inherited several limitations from UML, that limited it's expressiveness and usability for CPS modelling. However, it still provided a solid  foundation for specifying and analyzing a systems's behaviour, structure and requirements \cite{boelsen,cibrian,friedenthal,molnar,jansen}.

% (Shift) SysML v2 
The release of SysML v2 represents the next generation of the Systems Modeling Language, designed as a comprehensive overhaul of SysML v1 that address it's limitations and enhance the efficacy of MBSE practices.
Unlike it's predecessor, SysML v2 is built upon the Kernel Modeling Language (KerML), this approach ensures that SysML v2 inherits a formal semantic foundation, that is crucial for enhanced percision and automation in MBSE workflows.
\cite{boelsen,cibrian,friedenthal,molnar,jansen,ratzke}.

% Problem Statement (SysML v1 as a standard, its limitations, and SysML v2 as a solution)
While these advancements introduced by SysML v2 are promising, the abstract nature of the language still presents challenges for ensuring consistent modeling practices across diverse engineering teams.
Model inconsistencies creates a high risk of redundant effort, potential modeling errors, and lack of reuse of system elements, preventing them from being aggregated into a coherent overall system model.
Therefore, mechanisms for implementing and enforcing modeling guidelines must evolve to leverage the native formal capabilities of SysML v2, that enable effective model verification processes throughout the development lifecycle \cite{boelsen,bergemann,cibrian,molnar,ratzke}.

% Outline the difference between verification and validation
Within the context of model analysis, a distinct differentiation between model validation and model verification is necessary. 

\textit{Model validation} is a form of static analysis that ensures a model's structural and semantic correctness. By detecting inconsistencies ranging from syntatic errors to mismatched data types, validation guarantees the model's precision and coherence. This process as a critical quality control, ensuring the model is reliable before it undergoes transformation or dynamic verification \cite{cibrian, zavada}.

\textit{Model verification}, conversely, is a dynamic analysis technique focused on ensuring that the system specified by the model behaves exactly as intended under various conditions, thereby confirming behavioral adherence to specifications \cite{zavada}.

% Research Question & Thesis Statement & Methodology of the thesis
%% Highlight the focus which is on specific scenarios not a holistic overview of the tools
This work investigates the difference between mechanisms for implementing modeling guidelines in SysML v1 and SysML v2.
We analyze the implications these differences have on model verification processes, by looking into a couple of specific scenarios.

%
% Theoretical Background
%
\section{Theoretical Background}
\subsection{Model-Based Systems Engineering (MBSE)}
Model-Based Systems Engineering (MBSE) is defined by the International Council on Systems Engineering (INCOSE) as ''The formalized application of modeling to support system requirements, design, analysis, verification, and validation activities beginning in the concept stage and continuing throughout development and later life cycle stages.'' \cite{incose}. 
Unlike traditional document-based approaches, where system's information is scattered across various files, MBSE is a methodology the centralizes system's information within a unified model, which serves as the primary source of information that is managed throughout the system lifecycle \cite{friedenthal}.
This approach enhances collaboration among multidisciplinary teams, reduces errors from manual synchronization, and enables system architects to respond more quickly and effectively to numerous changes in requirements that occur during the development process \cite{dukic,yildirim,beers}.

\subsection{SysML v1 Foundations}

\subsection{SysML v2 Foundations}
% molnar

\clearpage
%
% Results
%
\section{Results}
\subsection{SysML v1 Implementation of Modeling Guidelines}
\subsection{SysML v2 Implementation of Modeling Guidelines}
\subsection{Comparative Analysis}

%
% Discuss & Implications
%
\section{Discussion and Implications}

%
% Conclusion & Outlook
%
\section{Conclusion and Outlook}

%
% Acknowledgment
%
\section*{Acknowledgment}

%
% Bibliography
%
\begin{thebibliography}{00}
    \bibitem{beers} L. Beers, H. Nabizada, M. Weigand, F. Gehlhoff, and A. Fay, “A sysml profile for the standardized description of processes during system development,” in 2024 IEEE International Systems Conference (SysCon). IEEE, 2024, pp. 1--8.
    
    \bibitem{bergemann} S. Bergemann, “Challenges in multi-view model consistency management for systems engineering,” in Modellierung 2022 Satellite Events. Bonn: Gesellschaft f¨ur Informatik e.V., 2022, pp. 77--89.

    \bibitem{boelsen} K. Boelsen, M. May, G. Jacobs, and et al., “Sysml v2 based modelling guidelines for mechanical system elements,” Forsch Ingenieurwes, vol. 89, no. 60, 2025.
    
    \bibitem{cibrian} E. Cibri\'{a}n, J. Olivert-Iserte, C. D\'{i}ez-Fenoy, R. Mendieta, J. Llorens, and J. M. \'Alvarez Rodr\'{i}guez, “Ensuring semantic consistency in sysml v2 models through metamodel-driven validation,” IEEE Access, vol. 13, pp. 121 444--121 457, 2025.
    
    \bibitem{dehn} S. Dehn, G. Jacobs, P. H\"ock, and G. H\"opfner, “Enhancing model-based development with formalized requirements: integrating temporal logic and sysml v2 for comprehensive state and transition modeling,” Forschung im Ingenieurwesen, vol. 89, no. 1, p. 53, 2025.
    
    \bibitem{dukic} P. Duki\'c, “Generating sysmlv2 models from structured natural language,” 2025.
    
    \bibitem{jansen} N. Jansen, J. Pfeiffer, B. Rumpe, D. Schmalzing, and A. Wortmann, “The language of sysml v2 under the magnifying glass.” J. Object Technol., vol. 21, no. 3, pp. 3--1, 2022.
    
    \bibitem{kausch} H. Kausch, M. Pfeiffer, D. Raco, B. Rumpe, and A. Schweiger, “Model-driven development for functional correctness of avionics systems: a verification framework for sysml specifications,” CEAS Aeronautical Journal, vol. 16, no. 1, pp. 33--48, 2025.
    
    \bibitem{molnar} V. Moln\'{a}r, B. Graics, A. V\"or\"os, S. Tonetta, L. Cristoforetti, G. Kimberly, P. Dyer, K. Giammarco, M. Koethe, J. Hester et al., “Towards the formal verification of sysml v2 models,” in Proceedings of the ACM/IEEE 27th International Conference on Model Driven Engineering Languages and Systems, 2024, pp. 1086--1095.
    
    \bibitem{ratzke} A. Ratzke, J. Koch, and C. Grimm, “Modeling and analysis of system models with constraints in sysml v2,” in 2025 20th Annual System of Systems Engineering Conference (SoSE), 2025, pp. 1--6.
    
    \bibitem{friedenthal} S. Friedenthal, “Future directions for mbse with sysml v2.” in MODEL-SWARD, 2023, pp. 5--9.
    
    \bibitem{raedler} S. Raedler, J. Mangler, and S. Rinderle-Ma, “Model-driven engineering method to support the formalization of machine learning using sysml,” arXiv preprint arXiv:2307.04495, 2023.

    \bibitem{yildirim}  U. Yildirim, F. Campean, A. Korsunovs, and A. Doikin, “Flow heuristics for functional modelling in model-based systems engineering,” Proceedings of the Design Society, vol. 3, pp. 1895--1904, 2023.
    
    \bibitem{zavada} Zavada, G. Kulcs\'{a}r, V. Moln\'{a}r, and Horv\'{a}th, “Towards a configurable verification and validation framework for critical cyber-physical systems,” in 2025 IEEE 8th International Conference on Industrial Cyber-Physical Systems (ICPS), 2025, pp. 1--6.
    
    \bibitem{zirui} Z. Li, F. Faheem, and S. Husung, “Collaborative model-based systems engineering using dataspaces and sysml v2,” Systems, vol. 12, no. 1, 2024. [Online]. Available: https://www.mdpi.com/2079-8954/12/1/18

    \bibitem{incose}
\end{thebibliography}

\end{document}
